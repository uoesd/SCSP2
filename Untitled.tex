% Options for packages loaded elsewhere
\PassOptionsToPackage{unicode}{hyperref}
\PassOptionsToPackage{hyphens}{url}
\documentclass[
  11pt,
]{article}
\usepackage{xcolor}
\usepackage[margin=1in]{geometry}
\usepackage{amsmath,amssymb}
\setcounter{secnumdepth}{5}
\usepackage{iftex}
\ifPDFTeX
  \usepackage[T1]{fontenc}
  \usepackage[utf8]{inputenc}
  \usepackage{textcomp} % provide euro and other symbols
\else % if luatex or xetex
  \usepackage{unicode-math} % this also loads fontspec
  \defaultfontfeatures{Scale=MatchLowercase}
  \defaultfontfeatures[\rmfamily]{Ligatures=TeX,Scale=1}
\fi
\usepackage{lmodern}
\ifPDFTeX\else
  % xetex/luatex font selection
\fi
% Use upquote if available, for straight quotes in verbatim environments
\IfFileExists{upquote.sty}{\usepackage{upquote}}{}
\IfFileExists{microtype.sty}{% use microtype if available
  \usepackage[]{microtype}
  \UseMicrotypeSet[protrusion]{basicmath} % disable protrusion for tt fonts
}{}
\makeatletter
\@ifundefined{KOMAClassName}{% if non-KOMA class
  \IfFileExists{parskip.sty}{%
    \usepackage{parskip}
  }{% else
    \setlength{\parindent}{0pt}
    \setlength{\parskip}{6pt plus 2pt minus 1pt}}
}{% if KOMA class
  \KOMAoptions{parskip=half}}
\makeatother
\usepackage{graphicx}
\makeatletter
\newsavebox\pandoc@box
\newcommand*\pandocbounded[1]{% scales image to fit in text height/width
  \sbox\pandoc@box{#1}%
  \Gscale@div\@tempa{\textheight}{\dimexpr\ht\pandoc@box+\dp\pandoc@box\relax}%
  \Gscale@div\@tempb{\linewidth}{\wd\pandoc@box}%
  \ifdim\@tempb\p@<\@tempa\p@\let\@tempa\@tempb\fi% select the smaller of both
  \ifdim\@tempa\p@<\p@\scalebox{\@tempa}{\usebox\pandoc@box}%
  \else\usebox{\pandoc@box}%
  \fi%
}
% Set default figure placement to htbp
\def\fps@figure{htbp}
\makeatother
\setlength{\emergencystretch}{3em} % prevent overfull lines
\providecommand{\tightlist}{%
  \setlength{\itemsep}{0pt}\setlength{\parskip}{0pt}}
\usepackage{fancyhdr}
\pagestyle{fancy}
\fancyhead[L]{\textit{Short report title}}
\fancyhead[R]{\leftmark}
\fancyfoot[C]{\thepage}
\usepackage{setspace}
\onehalfspacing
\usepackage{titling}
\pretitle{\begin{center}\LARGE\bfseries}
\posttitle{\end{center}\vspace{1em}}
\usepackage{caption}
\captionsetup[figure]{labelfont=bf,textfont=it}
\captionsetup[table]{labelfont=bf}
\usepackage{booktabs}
\usepackage{longtable}
\usepackage{array}
\usepackage{multirow}
\usepackage{wrapfig}
\usepackage{float}
\usepackage{colortbl}
\usepackage{pdflscape}
\usepackage{tabu}
\usepackage{threeparttable}
\usepackage{threeparttablex}
\usepackage[normalem]{ulem}
\usepackage{makecell}
\usepackage{xcolor}
\usepackage{bookmark}
\IfFileExists{xurl.sty}{\usepackage{xurl}}{} % add URL line breaks if available
\urlstyle{same}
\hypersetup{
  pdftitle={Report Title: Example Report},
  hidelinks,
  pdfcreator={LaTeX via pandoc}}

\title{Report Title: Example Report}
\author{Group 3\\
Department / Course}
\date{November 16, 2025}

\begin{document}
\maketitle

{
\setcounter{tocdepth}{3}
\tableofcontents
}
\section{Executive Summary}\label{executive-summary}

In the UK it is a criminal offence to drive a motor vehicle with a blood
or breath alcohol concentration above the prescribed limit. When a
person is arrested for driving under the influence of alcohol it is not
usually possible to perform an accurate test of the level of alcohol in
the blood or breath immediately. Breath tests can be used as an initial
screening tool at the scene, but these are not sufficiently accurate for
prosecution. Instead, people are taken to a police station or hospital,
where the test can be carried out using proper laboratory protocols. As
the body clears alcohol from the blood through time this means that if
the individual was over the limit, the measured blood alcohol
concentration (BAC) will be lower at the time of measurement than it was
when the person was driving a motor vehicle. To deal with this
situation, If the BAC after time \(t\) (hours) is measured as \(C_t\)
(g/kg), the BAC at time \(0\) is estimated as \(C_0 = C_t - βt\), where
\(\beta\) (\(g/kg/h\)) is BAC elimination rate.

The key point is how to find a precise \(\beta\) to estimate \(C_0\).
Forensic scientists currently \(2.5\%\) percentile of \(\beta\)
distribution constructing from samples as the estimated \(\beta\) value
for every individuals. This method is obviously not rigorous enough for
the courts, since:

\begin{itemize}
\tightlist
\item
  The courts will be forced to make decisions under estimated \(\beta\)
  if we only give a single estimation of \(beta\), since the calculated
  \(C_0\) is either over or under the legal limit.
\item
  Differences between individuals are ignored, for example age and sex,
  which may affect \(\beta\).
\item
  \(\beta\) value at \(2.5\%\) is over conservative and most
  uncertainties are hiding.
\end{itemize}

In this report, we will introduce a Bayesian regression model with
considering the heterogeneity between individuals, the model gives a
posterior distribution of \(\beta\) by using both samples and prior
knowledge. Then we randomly simulate \(4000\) \(\beta\) values from
posterior distribution and find out the probability of the person's BAC
over limit while driving.

In real world, \(\beta\) estimation is quite difficult, it depends on
the individuals' liver condition, drinking habits, genetic, diet, etc.
As we don't have corresponding data, in our regression model of
\(\beta\) is mainly dominate by gender, but we still find some useful
variables:

\begin{itemize}
\item
  Gender: Female's BAC elimination rate is larger than male on average,
  it is explained by liver weight represents a greater fraction of lean
  body mass in the female gender {[}2{]}.
\item
  Drinking time after BAC peak: If the person keep drinking after BAC
  reaches peak, the measured \(\beta\) will be smaller since \(\beta\)
  is measured start at BAC peak time till the end.
\end{itemize}

When it is too late to use a blood or breath test and the only
information available is eyewitness testimony of the quantity of alcohol
consumed. We have to use Widmark's equation:
\[C_t = \frac{A}{Weight \times V_d} - \beta t\]

where \(A\) is Amount of Alcohol Consumed (g), \(V_d\) is the volume of
distribution that need to be found. Forensic scientists use the same
method again to estimate \(V_d\) separately with \(\beta\). But \(V_d\)
and \(\beta\) are not independent, so we build a joint Bayesian
regression model, which can simulate them together to solve with the
correlation. Then again after simulation, each \(C_t\) is calculated by
each pair of \(\beta\) and \(V_d\), so expert witness can still find a
probability of \(C_t\) exceeding the limit.

To make it easier, we write the method in a function so other expert
witness can also get the results by easily plugging new person's data.

\section{Data Description}\label{data-description}

It is very important to normalized all variables like weight and height.
Since \(\beta\) is a small value, the coefficients of large value
variables are pretty small, near 0, which will make \(\beta\) estimation
worse. The other important reason is centering the covariates can reduce
autocorrelation

\[\rho_k = \frac{Cov(\beta^i, \beta^{i+k})}{\sigma^2},\]

then the Effective Sample Size (ESS):

\[ n_{\text{eff}} = \frac{n}{1 + 2 \sum_{k=1}^{\infty}\rho_k}\]

is lager, which means the MCMC method converge faster (for example if we
set \(4\) chains with each chains \(4000\) iterations, \(1000\) burn-in,
the \(n_{\text{eff}} = 10000\) means the number of independent samples
is \(10000\) out of \(16000\)). Larger \(n_{\text{eff}}\) in same number
of iterations means smaller variance of estimated exoectation since:

\[\mathrm{V}(\hat{E}(\beta)) \approx \frac{\sigma^2}{n_{\text{eff}}}.\]

Then we prefer to \(\beta\) to \(\log(\beta)\) since the original
\(\beta\) distribution has heavy tail, all outliers (if have) can be
scaled. As shown in figure, \(\beta\) forms a good normal distribution
shape after log-transferring. After simulation we can take exponential
of the results.

Since \(\beta\) is a constant rate, measured when the BAC curve reaches
peak and starts to decrease, so variables like AAC and Maximum BAC will
not be used, since they are correlated with the value of BAC, not BAC
elimination rate.

From the dataset, most data comes from age smaller than \(30\)
(\(85/100\)), so even though age is sufficient variable for \(\beta\)
the model can't really recognize it. Research shows that age doesn't
really matter unless the subject has age-related liver disease, but
because of moral and ethical, we don't find much research of alcohol
test on elderly person with liver disease.

\section{Bayesian regression model
(BRM)}\label{bayesian-regression-model-brm}

\subsection{Reason for BRM}\label{reason-for-brm}

Comparing with fitting linear regression model for \(\beta_i\), which
can be expressed as:

\[\hat{\beta}_i = \hat{\gamma}_{1}\,\text{female}_i
+ \hat{\gamma}_{2}\,\text{male}_i
+ \hat{\gamma}_{3}\,\text{weight}_i
+ \hat{\gamma}_{4}\,\text{height}_i + \sigma^2\]

and giving a exact value of \(\hat{\beta}\), BRM can model uncertainty
from each coefficient \(\gamma\), since \(\beta\) from each individuals
may affect by variables in different level. We provide priors and sample
likelihoods for each variables, BRM will apply bayesian rule to each
variables and output simulation results of coefficients specifically for
each individual:

\[\hat{\beta}_i = \hat{\gamma}_{1,i}\,\text{female}_i
+ \hat{\gamma}_{2,i}\,\text{male}_i
+ \hat{\gamma}_{3,i}\,\text{weight}_i
+ \hat{\gamma}_{4,i}\,\text{height}_i + \sigma^2_i.\]

Since we have simulation results of \(\hat{\beta}\), we can offer a
probability for the courts.

\subsection{How BRM works}\label{how-brm-works}

We use \(brm\) package in R to do this.

The principle theorem behind BRM is Bayes' Rule:
\[p(\theta \mid \text{data}) = \frac{p(\text{data} \mid \theta) \, p(\theta)}{p(\text{data})}.\]

The \(\theta\) in the equation is the coefficients of \(\gamma_j\). For
most real models, we can't compute this posterior
\(p(\theta \mid \text{data})\) analytically. Markov Chain Monte Carlo
(MCMC) helps us draw samples from it, which we can then summarize. BRM
uses Hamiltonian Monte Carlo (One of MCMC), which is better than
standard MCMC (like Metropolis-Hastings) since it is faster and cost
less.

After setting priors for \(\gamma_j\), we need specify a family for
\(\beta\) as the the likelihood of the observed data. In our sample, we
have already taken \(\log(\beta)\), so it is better to use the most
common normal distribution likelihood, it is normal and can express
\(\beta\) values good without restrict samples' shape.

Also for the MCMC simulation method, we set \(4\) chains, each chains
\(2000\) draws with \(500\) burn-in draws. Since MCMC is a random walk
method through parameter space. One chain is a single run of the MCMC
algorithm that generates a sequence of samples, so we need more chains
to check the convergence and ensure that all chains convergent to same
mode.

For \(500\) burn-in draws, since MCMC starts from an initial value which
often far from the true posterior region, so the chains need some steps
to walk to the correct trace. As Figure 1 showed, it is an example of
the MCMC process of `sexfemale' when using \(4\) chains, \(100\) draws.
Clearly that if we keep the first \(50\) draws, the results will be
affected.

\subsection{Model Selection}\label{model-selection}

\subsubsection{Key Variables}\label{key-variables}

\subsubsection{Priors}\label{priors}

Before fitting the Bayesian regression model, appropriate prior
distributions need to be specified for all regression coefficients and
the residual standard deviation. We use weakly informative priors so
that the observed data dominate the inference process. The selected
prior distributions are summarized as follows:

All variables have been standardised, so choosing Normal(0,·) as the
weak information prior is reasonable. Based on the review by Jones
(2010), sex is considered one of the primary factors influencing alcohol
elimination rates. Therefore, we set a relatively wide prior Normal(0,
2) for the sex coefficients, allowing their effects to vary across a
broad and reasonable range without being overly constrained by the
prior. For other standardized variables (such as weight and height),
since their impact on β is considered relatively small, we used a more
shrinkage weakly informative prior Normal(0, 0.5) to prevent
unreasonably large effects. The residual standard deviation is given an
exponential(1) prior, which is a commonly used positive weakly
informative prior that can avoid unreasonably high noise.

If future research involves different populations, larger sample sizes,
or additional biological information, the prior distributions can be
adjusted accordingly to ensure they are applicable to any new dataset.

reference: Jones, A. W. (2010). Evidence-based survey of the
elimination rates of ethanol from blood with applications in forensic
casework. Forensic Science International, 200, 1--20.
\url{https://doi.org/10.1016/j.forsciint.2010.02.021}

\subsection{Model Results}\label{model-results}

\subsubsection{Fitting result}\label{fitting-result}

After \(4000\) iterations in \(4\) chains, Figure 3 shows posterior
results for each variables' coefficients, all coefficients show good
normal patterns since we all chains have converged and warm-up numbers
is enough. Since the \(\beta\) is log-transformed, the value of
`b\_sexfemale' and `b\_sexmale' is not typical value of \(\beta\) like
\(0.19\) when other coefficients are all near zero.

\subsubsection{PPC Density}\label{ppc-density}

Figure 3 is the Posterior Predictive Check (PPC) density plot comparing
with true value of \(\beta\) (Here we have taken exponential of both
true value and predicted value). Notice that predicted posterior
distribution matches \(\beta\) sample distribution pretty well, which
get the mean, spread, skew both right even at right tail.

\subsection{Model Testing}\label{model-testing}

\subsubsection{LOO-PIT QQ plot}\label{loo-pit-qq-plot}

The Leave-One-Out Probability Integral Transform Quantile-Quantile plot
(LOO-PIT QQ plot) can check if the predictions calibrated correctly and
if the predictive distributions biased, which is defined as:
\[ \mathrm{LOO\text{-}PIT}_i = P(\tilde{y}_i \le y_i |y_{-i}) .\]

Here we still use Monte Carlo method to simulate it, as:

\[
\mathrm{LOO\text{-}PIT}_i
= \frac{1}{S} \sum_{s=1}^{S} 
\mathbf{1}\!\left( \tilde{y}_{i,-i}^{(s)} \le y_i \right)
\] where \(S\) is the number of posterior draws (we set it 10000 here).
We will use it to check if our posterior is under or over dispersion,
the good model's PIT plot should be flat over \(Uniform(0,1)\). For our
model, the PIT histogram (Figure 4) is flat everywhere except
probability near 0.55, which is acceptable by the randomness of MCMC
method. A U-shape (points over dot line at two tails) in PIT plots means
overestimates variance, it doesn't appear in our plot gives evidence of
our choices on variables and likelihood family.

\section{New method: build a linear regression model for
beta}\label{new-method-build-a-linear-regression-model-for-beta}

Check if age, weight and height have linear relationships with beta. Or
have any patterns?

\section{Calculate V\_d:}\label{calculate-v_d}

in the Widmark's equation assumption, beta and V\_d are independent, so
we need to check whether Cov(V\_d, beta) = 0

\section{Further research:}\label{further-research}

casual effect: biased data selection, (e.g.~inner correlations between
sex and height)

\subsection{REPORT}\label{report}

\section{Exclusive summary}\label{exclusive-summary}

background Goal - 1, 2, 3 main data + elimination rate explanation
method - model we used result

\section{Overview of Datasets}\label{overview-of-datasets}

\subsection{Data collection}\label{data-collection}

\subsection{Variable Description}\label{variable-description}

\begin{itemize}
\tightlist
\item
  how many observations do we have? where we get the resourse,
  reference? explain each variables in table
\end{itemize}

The variables identified for the analysis are demographic and
physiological characteristics from the tested individuals after drinking
alcohol. Our variable selection is informed by their established
relevance to human liver metabolic function, especially sex, age, weight
and height are included due to the significant influence on liver
metabolism.

\begin{itemize}
\tightlist
\item
  Sex: Biological sex of the individual, male or female
\item
  Age:
\item
  Weight\\
\item
  Height
\item
  Beta60:
\item
  Co:
\end{itemize}

Page, Participant number, FigureOnPage, Sample Type\ldots{} clearly
inrevelant,

Maximum BAC/BrAC,BAC peak time 也可推断为无关factor
与police预测受测者驾驶中酒精浓度的目标不同

\subsection{Prelimary analysis}\label{prelimary-analysis}

\begin{itemize}
\tightlist
\item
  initial formula: explain beta60 + - motivation to improve?
\end{itemize}

variable correlations? Why? hypothesis needed? why 2.5 quantiles?

\section{Model selection regarding key variable
`beta60'}\label{model-selection-regarding-key-variable-beta60}

\subsection{Advantage of current
method}\label{advantage-of-current-method}

simpler model with only one parameter(beta), easier to calculate,
suitable for no dataset situations

\subsection{New model motivation}\label{new-model-motivation}

Current model disadvantage: 1. beta is fixed for all individuals, we
need to consider the heterogeneity in individuals. 2. Deviated aim:
Police Scotland aims at assessing the probabilities that a person is
over the limit, while the parameter beta only concentrates on the range
between 2.5\% and 97.5\%.

\section{Bayesian model}\label{bayesian-model}

intro New model (Bayesian) \#\# Model Selection and Evaluation - why
Bayesian? Any advantage, any improvement? More accurate? \ldots{}
\#\#explanation \#\#\#why prior setted? \#\# how result reflect?

\subsection{Model fitting example}\label{model-fitting-example}

How our new model works on the 70 year-old madam?

\section{Testing another
assumption(condition):}\label{testing-another-assumptioncondition}

\subsection{variable explanation}\label{variable-explanation}

explain all variables we used explain needed references

\#\#Testing whether formula reasonable? whether beta60 relate to Vt? how
related? hows the result? how new model fit?

\#Limitation our model limit data collected reason - cannot reflect the
real world model reason other reason: variables

\#N

\#Reference

空腹低 长期饮酒高CTP2E1活性高 性别肝脏大小 药物抑制酒精代谢
测两次血预测beta
如果司机坐在方向盘后面没有被捕,瑞典的交警通常会提交间隔约1小时采集的双重血液样本。假设存在吸收后下降阶段和零阶动力学操作,每个采样时间的平均结果可用于计算酒精的消除率。根据长期调查酒后驾车案件的经验,第一个血液样本通常在逮捕后60分钟获得,这取决于该人被逮捕的全国地点以及是否有医生或护士抽血。
\url{https://www.sciencedirect.com/science/article/pii/S0379073810000770?via\%3Dihub}

P.Y. Kwo, V.A. Ramchandani, S. O'Connor, D. Amann, L.G. Carr, K.
Sandrasegaran, K.K. Kopecky, T.K. Li Gender differences in alcohol
metabolism: relationship to liver volume and effect of adjusting for
body mass Gastroenterology, 115 (1998), pp.~1552-1557

\end{document}
