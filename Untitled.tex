% Options for packages loaded elsewhere
\PassOptionsToPackage{unicode}{hyperref}
\PassOptionsToPackage{hyphens}{url}
\documentclass[
  11pt,
]{article}
\usepackage{xcolor}
\usepackage[margin=1in]{geometry}
\usepackage{amsmath,amssymb}
\setcounter{secnumdepth}{5}
\usepackage{iftex}
\ifPDFTeX
  \usepackage[T1]{fontenc}
  \usepackage[utf8]{inputenc}
  \usepackage{textcomp} % provide euro and other symbols
\else % if luatex or xetex
  \usepackage{unicode-math} % this also loads fontspec
  \defaultfontfeatures{Scale=MatchLowercase}
  \defaultfontfeatures[\rmfamily]{Ligatures=TeX,Scale=1}
\fi
\usepackage{lmodern}
\ifPDFTeX\else
  % xetex/luatex font selection
\fi
% Use upquote if available, for straight quotes in verbatim environments
\IfFileExists{upquote.sty}{\usepackage{upquote}}{}
\IfFileExists{microtype.sty}{% use microtype if available
  \usepackage[]{microtype}
  \UseMicrotypeSet[protrusion]{basicmath} % disable protrusion for tt fonts
}{}
\makeatletter
\@ifundefined{KOMAClassName}{% if non-KOMA class
  \IfFileExists{parskip.sty}{%
    \usepackage{parskip}
  }{% else
    \setlength{\parindent}{0pt}
    \setlength{\parskip}{6pt plus 2pt minus 1pt}}
}{% if KOMA class
  \KOMAoptions{parskip=half}}
\makeatother
\usepackage{graphicx}
\makeatletter
\newsavebox\pandoc@box
\newcommand*\pandocbounded[1]{% scales image to fit in text height/width
  \sbox\pandoc@box{#1}%
  \Gscale@div\@tempa{\textheight}{\dimexpr\ht\pandoc@box+\dp\pandoc@box\relax}%
  \Gscale@div\@tempb{\linewidth}{\wd\pandoc@box}%
  \ifdim\@tempb\p@<\@tempa\p@\let\@tempa\@tempb\fi% select the smaller of both
  \ifdim\@tempa\p@<\p@\scalebox{\@tempa}{\usebox\pandoc@box}%
  \else\usebox{\pandoc@box}%
  \fi%
}
% Set default figure placement to htbp
\def\fps@figure{htbp}
\makeatother
\setlength{\emergencystretch}{3em} % prevent overfull lines
\providecommand{\tightlist}{%
  \setlength{\itemsep}{0pt}\setlength{\parskip}{0pt}}
\usepackage{fancyhdr}
\pagestyle{fancy}
\fancyhead[L]{\textit{Short report title}}
\fancyhead[R]{\leftmark}
\fancyfoot[C]{\thepage}
\usepackage{setspace}
\onehalfspacing
\usepackage{titling}
\pretitle{\begin{center}\LARGE\bfseries}
\posttitle{\end{center}\vspace{1em}}
\usepackage{caption}
\captionsetup[figure]{labelfont=bf,textfont=it}
\captionsetup[table]{labelfont=bf}
\usepackage{booktabs}
\usepackage{longtable}
\usepackage{array}
\usepackage{multirow}
\usepackage{wrapfig}
\usepackage{float}
\usepackage{colortbl}
\usepackage{pdflscape}
\usepackage{tabu}
\usepackage{threeparttable}
\usepackage{threeparttablex}
\usepackage[normalem]{ulem}
\usepackage{makecell}
\usepackage{xcolor}
\usepackage{bookmark}
\IfFileExists{xurl.sty}{\usepackage{xurl}}{} % add URL line breaks if available
\urlstyle{same}
\hypersetup{
  pdftitle={Report Title: Example Report},
  hidelinks,
  pdfcreator={LaTeX via pandoc}}

\title{Report Title: Example Report}
\author{Group 3\\
Department / Course}
\date{November 15, 2025}

\begin{document}
\maketitle

{
\setcounter{tocdepth}{2}
\tableofcontents
}
\section{Executive Summary}\label{executive-summary}

In the UK it is a criminal offence to drive a motor vehicle with a blood
or breath alcohol concentration above the prescribed limit. When a
person is arrested for driving under the influence of alcohol it is not
usually possible to perform an accurate test of the level of alcohol in
the blood or breath immediately. Breath tests can be used as an initial
screening tool at the scene, but these are not sufficiently accurate for
prosecution. Instead, people are taken to a police station or hospital,
where the test can be carried out using proper laboratory protocols. As
the body clears alcohol from the blood through time this means that if
the individual was over the limit, the measured blood alcohol
concentration(BAC) will be lower at the time of measurement than it was
when the person was driving a motor vehicle. To deal with this
situation, If the BAC after time \(t\) (hours) is measured as \(C_t\)
(g/kg), the BAC at time \(0\) is estimated as \(C_0 = C_t - βt\), where
\(\beta\) (\(g/kg/h\)) is BAC elimination rate.

The key point is how to find a precise \(\beta\) to estimate \(C_0\).
Forensic scientists currently \(2.5\%\) percentile of \(\beta\)
distribution constructing from samples as the estimated \(\beta\) value
for every individuals. This method is obviously not rigorous enough for
the courts since:

\begin{itemize}
\tightlist
\item
  The courts will be forced to make decisions under estimated \(\beta\)
  if we only give a single estimation of \(beta\), since the calculated
  \(C_0\) is either over or under the legal limit.
\item
  Differences between individuals are ignored, for example age and sex,
  which may affect \(\beta\).
\item
  \(\beta\) value at \(2.5\%\) is over conservative.
\end{itemize}

\section{Motivations: Need improve?
Why?}\label{motivations-need-improve-why}

\section{Disadvantage of current
method}\label{disadvantage-of-current-method}

\begin{enumerate}
\def\labelenumi{\arabic{enumi}.}
\tightlist
\item
  beta is fixed for all individuals, we need to consider the
  heterogeneity in individuals.
\item
  Deviated aim: Police Scotland aims at assessing the probabilities that
  a person is over the limit, while the parameter beta only concentrates
  on the range between 2.5\% and 97.5\%.
\end{enumerate}

\section{New method: build a linear regression model for
beta}\label{new-method-build-a-linear-regression-model-for-beta}

Check if age, weight and height have linear relationships with beta. Or
have any patterns?

\section{Model selection:}\label{model-selection}

use drop(), look into AIC and R\^{}2

\section{Hypothesis test(70yr female example-better use 2.5th
quantile?):}\label{hypothesis-test70yr-female-example-better-use-2.5th-quantile}

H0: C0 \textless{} limit, H1: C0 \textgreater{} limit

\section{Calculate V\_d:}\label{calculate-v_d}

in the Widmark's equation assumption, beta and V\_d are independent, so
we need to check whether Cov(V\_d, beta) = 0

\section{Further research:}\label{further-research}

casual effect: biased data selection, (e.g.~inner correlations between
sex and height)

\subsection{REPORT}\label{report}

\section{Exclusive summary}\label{exclusive-summary}

background Goal - 1, 2, 3 main data + elimination rate explanation
method - model we used result

\section{Overview of Datasets}\label{overview-of-datasets}

\subsection{Variable Description}\label{variable-description}

\begin{itemize}
\tightlist
\item
  how many observations do we have? where we get the resourse,
  reference? explain each variables in table \#\# Prelimary analysis
\item
  initial formula: explain beta60 + - motivation to improve?
\end{itemize}

variable correlations? Why? hypothesis needed? why 2.5 quantiles?

\section{Model selection regarding key variable
`beta60'}\label{model-selection-regarding-key-variable-beta60}

\subsection{Advantage of current
method}\label{advantage-of-current-method}

simpler model with only one parameter(beta), easier to calculate,
suitable for no dataset situations

\subsection{New model motivation}\label{new-model-motivation}

Current model disadvantage: 1. beta is fixed for all individuals, we
need to consider the heterogeneity in individuals. 2. Deviated aim:
Police Scotland aims at assessing the probabilities that a person is
over the limit, while the parameter beta only concentrates on the range
between 2.5\% and 97.5\%.

\section{Bayesian model}\label{bayesian-model}

intro New model (Bayesian) \#\# Model Selection and Evaluation - why
Bayesian? Any advantage, any improvement? More accurate? \ldots{}
\#\#explanation \#\#\#why prior setted? \#\# how result reflect?

\subsection{Model fitting example}\label{model-fitting-example}

How our new model works on the 70 year-old madam?

\section{Testing another
assumption(condition):}\label{testing-another-assumptioncondition}

\subsection{variable explanation}\label{variable-explanation}

explain all variables we used explain needed references

\#\#Testing whether formula reasonable? whether beta60 relate to Vt? how
related? hows the result? how new model fit?

\#Limitation our model limit data collected reason - cannot reflect the
real world model reason other reason: variables

\end{document}
